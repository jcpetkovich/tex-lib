% macros.tex -- holds useful macros, acronyms, and other latex tricks

% Macros
\usepackage{xspace}
\newcommand\code[1]{\textsf{\small #1}\xspace}
\newcommand{\ie}{i.e.,\xspace}
\newcommand{\eg}{e.g.,\xspace}
\newcommand{\etal}{{\it et al.}\xspace}

% Acronyms
\usepackage{acronym}
\acrodef{os}[OS]{Operating System}
\acrodef{aslr}[ASLR]{Address Space Layout Randomization}
\acrodef{api}[API]{Application Programming Interface}
\acrodef{pcb}[PCB]{Process Control Block}
\acrodef{sp}[SP]{Stack Pointer}
\acrodef{pc}[PC]{Program Counter}
\acrodef{ipc}[IPC]{Inter-Process Communication}
\acrodef{pid}[PID]{Process ID}

% Tikz
\usepackage{tikz}
\tikzstyle{na} = [baseline=-.5ex]

% Set madhax one to disable:
\def\madhax{1}
\if\madhax 1
\makeatletter
\def\grd@save@target#1{%
  \def\grd@target{#1}}
\def\grd@save@start#1{%
  \def\grd@start{#1}}
\tikzset{
  grid with coordinates/.style={
    to path={%
      \pgfextra{%
        \edef\grd@@target{(\tikztotarget)}%
        \tikz@scan@one@point\grd@save@target\grd@@target\relax
        \edef\grd@@start{(\tikztostart)}%
        \tikz@scan@one@point\grd@save@start\grd@@start\relax
        \draw[minor help lines] (\tikztostart) grid (\tikztotarget);
        \draw[major help lines] (\tikztostart) grid (\tikztotarget);
        \grd@start
        \pgfmathsetmacro{\grd@xa}{\the\pgf@x/1cm}
        \pgfmathsetmacro{\grd@ya}{\the\pgf@y/1cm}
        \grd@target
        \pgfmathsetmacro{\grd@xb}{\the\pgf@x/1cm}
        \pgfmathsetmacro{\grd@yb}{\the\pgf@y/1cm}
        \pgfmathsetmacro{\grd@xc}{\grd@xa + \pgfkeysvalueof{/tikz/grid with coordinates/major step}}
        \pgfmathsetmacro{\grd@yc}{\grd@ya + \pgfkeysvalueof{/tikz/grid with coordinates/major step}}
        \foreach \x in {\grd@xa,\grd@xc,...,\grd@xb}
        \node[anchor=north] at (\x,\grd@ya) {\pgfmathprintnumber{\x}};
        \foreach \y in {\grd@ya,\grd@yc,...,\grd@yb}
        \node[anchor=east] at (\grd@xa,\y) {\pgfmathprintnumber{\y}};
      }
    }
  },
  minor help lines/.style={
    help lines,
    step=\pgfkeysvalueof{/tikz/grid with coordinates/minor step}
  },
  major help lines/.style={
    help lines,
    line width=\pgfkeysvalueof{/tikz/grid with coordinates/major line width},
    step=\pgfkeysvalueof{/tikz/grid with coordinates/major step}
  },
  grid with coordinates/.cd,
  minor step/.initial=.2,
  major step/.initial=1,
  major line width/.initial=2pt,
}
\makeatother
\fi

% For drawing grids, use the following in a tikz picture:
% \draw (0,0) to[grid with coordinates] (11,6);

% \highlight<on slide><but not these slides>[this text]. The first
% triangle arg of highlight defines the slide that the text should be
% highlighted on, the second triangle bracket contains the slides that
% it should not be highlighted on. The square brackets house the text.
%
% TODO: Is it possible to make it so that all the "rest" of the slides
% are added automatically? I'm not really good enough at latex to know
% if this is possible.
\def\highlight<#1><#2>[#3]{\only<#1>{\textcolor{datamillorange}{#3}}\only<#2>{#3}}